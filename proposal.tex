% Created 2018-09-16 Sun 15:27
% Intended LaTeX compiler: pdflatex
\documentclass[11pt]{article}
\usepackage[utf8]{inputenc}
\usepackage[T1]{fontenc}
\usepackage{graphicx}
\usepackage{grffile}
\usepackage{longtable}
\usepackage{wrapfig}
\usepackage{rotating}
\usepackage[normalem]{ulem}
\usepackage{amsmath}
\usepackage{textcomp}
\usepackage{amssymb}
\usepackage{capt-of}
\usepackage{hyperref}
\author{Kai Lukowiak}
\date{\today}
\title{Gas Price Micro Markets}
\hypersetup{
 pdfauthor={Kai Lukowiak},
 pdftitle={Gas Price Micro Markets},
 pdfkeywords={},
 pdfsubject={},
 pdfcreator={Emacs 26.1 (Org mode 9.1.9)}, 
 pdflang={English}}
\begin{document}

\maketitle
\begin{abstract}

Gas station prices are dependent on competiton. Due to the nature of gasoline
markets this competition is very local. Industry is not very sophisticated in
segmenting these micro-markets leaving money on the table.

This proposal shows the roadmap to use variation and similarity in regional prices
and geospatial data to identify these micro-markets and find outliers that are
close to but otherwise perform differently. 

The end goal is to provide information on these stations along with a framework
that companies can use to experiment further.

\end{abstract}

\section{Description of the Problem}
\label{sec:orgad65e28}
Retail Gas Prices (RGP) are determined in the competitive market. Gas stations
are surprisingly unsophisticated given that the market in Canada is \href{https://www.ibisworld.ca/industry-trends/market-research-reports/retail-trade/gas-stations.html}{\$34 billion}.
If pricing is granular to the site level, it is almost entirely set by regional
or even station managers and national campaigns do not have the granularity to
capitalize on market differences.

There are many facets that go into the definition of a competitive market.
Cataloguing all features and recording them is a gargantuan task. Instead this
study hopes to discern which stations may be different from their surrounding
competition. This will allow more targeted experimentation.



The competition is based on many factors like loyalty programs, marketing,
customer preferences etc. One major influence on RGP is geographic
location. Even within the same distribution market (same base fuel price), a
station on a major highway, one in close proximity to three other competitors
and one in a rural area will all behave differently.


\section{Why the problem is interesting}
\label{sec:orgbcc4749}

There is aproximetly \href{https://www150.statcan.gc.ca/t1/tbl1/en/tv.action?pid=2310006601}{\$50 billion L} of motor fuels sold in Canada yearly.
\(\50,000,000,000L \cdot \alpha\) where \(\alpha\) is small increase in cents per
litre (CPL) is still a lot of money. 

Beyond the fiscal incentives, this is an interesting study because it will
reveal (hopefully) two findings. The first is finding relevant markets in
geographic areas. For example, it is easy to believe that gas stations on
different sides of a major river or other barrier would price differently.

Secondly, after geographic markets are found, gas stations that are contained
within a similar geographic area, but that price differently can be further
analyzed. For example, a suburban area may be a well defined market with similar
price fluctuation in stations except for one. This will allow an analyst to
further investigate this one station to find which features make it more
profitable (or less).

These findings could help set prices more efficiently as well as indicate which
areas are best for future investment. 


\section{What other approaches have been tried}
\label{sec:org11599b1}

Investigating retail gasoline markets is nothing new. Most recently in Canada,
\href{https://www.parkland.ca/en/our-businesses/retail/}{Parkland} has a large market of retail stations and merged with \href{https://en.wikipedia.org/wiki/Pioneer\_Energy}{Pioneer Energy}
based in Ontario.  The \href{http://www.competitionbureau.gc.ca/eic/site/cb-bc.nsf/eng/04053.html}{Competition Bureau} forced Parkland to divest several
stations they believed would form regional monopolies.

I am also sure marketing departments have segmented promotional efforts based on
location.

Both of these focus on a more macro level geography.

There are many examples of spatial/temporal regression investigating gas prices
in the economic literature. Some of these are quite granular, although I was
unable to find any on the same scale as I plan. These are listed in the links
at the bottom of this proposal.

Further these look at the causal effect of distance on price. My intention is to
cluster sites and compare them to other clustering techniques such as
KNN. Some examples of spatial regression can be seen below as well.


\section{Discussion on your hypothesis is and how your specific solution will improve}
\label{sec:org9e342aa}

My hypothesis is that there are anomalous gas stations and unobserved
sub-regions which can be detected through variance spatial/temporal pricing.

Further, due to my disinterest in causality/interpretation I can use also use
non-linear techniques that make interpretation difficult.

My first approach will be to use \href{https://www.earthdatascience.org/tutorials/intro-to-spatial-regression/}{PySal} to investigate linear relationships
between stores and cluster them by similarity to their prices.

I also plan on investigating other correlation/regression techniques with
numbers weighted by distance.

After getting a functioning model, I plan on comparing this to clustering
algorithms such as KNN with a similar number of centroids to the previous model.
I will then compare the overlap to the model taking into account price. This
geographic comparison will shed light onto unique gas stations.

Further, this model can be adopted by industry to perform experiments on
individual stations allowing for the discovery of price leaders and even more
granular market segments.

\subsection{Effects of Edgeworth Cycles}
\label{sec:orgcf1686a}

\href{https://en.wikipedia.org/wiki/Edgeworth\_price\_cycle}{Edgeworth Price Cycles} occur in competitive industries where firms find it
profit maximizing to cut prices until they reach marginal cost. At this point,
they can profit maximize by increases prices well above marginal cost. This
cycle repeats, and in the case of retail gas, often on a weekly basis. I plan on
using this cycle to determine relevant markets. 

\subsection{Additional Features}
\label{sec:org6637e38}
I also plan on taking in historic prices for the area on an anualized basis.
This, along with weather, holiday and macroeconomic information will be added
to try and remove omitted variable bias.

\subsection{Data Acquisition}
\label{sec:org7030e06}

I plan on using a scraping script, AWS Lambda and S3 to find current gas prices
in Western Canada. Other data like historic weather conditions and gas prices,
will be found from Stats Can. 

Locational data will be scraped from google maps API.


\section{Additional Info}
\label{sec:orgae35d99}

\begin{itemize}
\item \url{https://link.springer.com/article/10.1007/s00168-007-0206-7}
\item \url{https://www.jstor.org/stable/20111978?read-now=1\&googleloggedin=true\&seq=1\#page\_scan\_tab\_contents}
\item \url{https://link.springer.com/chapter/10.1007/978-3-7908-2070-6\_12}
\item \url{http://journals.ama.org/doi/abs/10.1509/jmkr.44.4.622?code=amma-site}
\item \url{https://www.jstor.org/stable/41323223?seq=1\%23page\_scan\_tab\_contents}
\item \url{http://ses.wsu.edu/wp-content/uploads/2015/03/SpatialDifferences.pdf}
\item \url{http://www.econ.uiuc.edu/\~lab/workshop/Spatial\_in\_R.html}
\item \url{http://darribas.org/gds\_scipy16/ipynb\_md/08\_spatial\_regression.html}
\end{itemize}
\end{document}
